\documentclass[12pt, ngerman]{beamer}

\mode<presentation> {}

% Import custom footer
\newif\ifshowpagenumbers
\showpagenumberstrue

\defbeamertemplate{footline}{customfooter}
{%
  \nointerlineskip%
  \leavevmode%
  \hbox{%
  \begin{beamercolorbox}[wd=.333333\paperwidth,ht=2.25ex,dp=1ex,sep=0pt,center]{author in head/foot}%
    \usebeamerfont{author in head/foot}Jakob Everding
  \end{beamercolorbox}%
  \begin{beamercolorbox}[wd=.333334\paperwidth,ht=2.25ex,dp=1ex,sep=0pt,center]{title in head/foot}%
    \usebeamerfont{title in head/foot}Example slides%
  \end{beamercolorbox}%
  \ifshowpagenumbers%
    \begin{beamercolorbox}[wd=.222222\paperwidth,ht=2.25ex,dp=1ex,sep=0pt,right]{date in head/foot}%
    \usebeamerfont{date in head/foot}UHH, 2025-09-25%
  \end{beamercolorbox}%
  \begin{beamercolorbox}[wd=.111111\paperwidth,ht=2.25ex,dp=1ex,sep=0pt,center]{date in head/foot}%
    \usebeamerfont{date in head/foot}\insertframenumber{} / \inserttotalframenumber\hspace*{2ex}%
  \end{beamercolorbox}%
  \else%
    \begin{beamercolorbox}[wd=.333333\paperwidth,ht=2.25ex,dp=1ex,sep=0pt,center]{date in head/foot}%
        \usebeamerfont{date in head/foot}UHH, 2025-09-25%
    \end{beamercolorbox}%
  \fi%
  \vskip0pt%
}%
}
\makeatother


\newcommand{\usecustomfooter}{
    \setbeamertemplate{footline}[customfooter]
}


\usepackage[ngerman, english]{babel}
\usepackage[iso]{datetime2}
\usepackage[ansinew]{inputenc}
\usepackage{empheq,pifont}
\usepackage{bbm} %With the bbm package (\usepackage{bbm}) it is possible to add symbols for several standard sets.
\usepackage[font=scriptsize, labelfont=scriptsize]{caption}
\usepackage{amsmath}
	\DeclareMathOperator*{\argmin}{arg\,min}
\usepackage{hyperref}
\usepackage[style=chicago-authordate, sorting=nyt, backend=bibtex, natbib=true, isbn=false, doi=false, maxbibnames=99, maxcitenames=2,uniquelist=false]{biblatex}
\usepackage{wrapfig}
\usepackage{array}
 \usepackage{appendixnumberbeamer}
\usepackage{appendix}
\usepackage{setspace}
\usepackage{graphicx, float, booktabs}
\usepackage{threeparttablex}	%combine threeparttable und longtable
\usepackage{threeparttable}
\usepackage{booktabs}
\usepackage{epstopdf}
\usepackage{longtable}
\usepackage{setspace}
\usepackage{siunitx}
\usepackage{rotating}
\usepackage{lmodern}
\usepackage{arydshln}
\usepackage[]{threeparttable}
\usepackage{dcolumn}
\usepackage{array}
\usepackage{multicol}
\setbeamersize{text margin left=5mm, text margin right=5mm}
\setbeamerfont{page number in head/foot}{size=\tiny}
\setbeamercovered{transparent=10}	%Gray overlay text
\setbeamertemplate{navigation symbols}{} %Supresses navigation bar

\setbeamercolor{structure}{fg=darkred} %% Defines the color of all items. Including all types of bulllet points and buttons.
%%\setbeamertemplate{itemize item}{\color{yellow}$\bullet$} %% Another way, but only for the bulletpoint
%%\setbeamertemplate{itemize subitem}{\color{green}$\blacktriangleright$} %% Another way, but only fpr the triangle

\setbeamertemplate{itemize item}[circle]
\setbeamertemplate{itemize subitem}[triangle]

\setbeamertemplate{caption}[numbered] %% Tables and Figures are enumerated automatically.
\usepackage{hanging}
\usepackage{tikz}

% Apply custom footer
\usecustomfooter
\showpagenumberstrue % Default is also true, so this is optional


%%The whole paragrapgh is necessary to decide whether a slide should have a bullet point in the header or not. \miniframeson and \miniframesoff are the commands for that.
\makeatletter
\let\beamer@writeslidentry@miniframeson=\beamer@writeslidentry
\def\beamer@writeslidentry@miniframesoff{%
  \expandafter\beamer@ifempty\expandafter{\beamer@framestartpage}{}% does not happen normally
  {%else
    % removed \addtocontents commands
    \clearpage\beamer@notesactions%
  }
}
\newcommand*{\miniframeson}{\let\beamer@writeslidentry=\beamer@writeslidentry@miniframeson}
\newcommand*{\miniframesoff}{\let\beamer@writeslidentry=\beamer@writeslidentry@miniframesoff}
\makeatother



\usecolortheme{beaver}
\usepackage{multirow}
\usepackage{bigdelim}
\renewcommand*{\familydefault}{\sfdefault}
\renewcommand\textbullet{\ensuremath{\bullet}}
\newcommand*{\head}{\bfseries}
\newcolumntype{_}{>{\global\let\currentrowstyle\relax}}
\newcolumntype{^}{>{\currentrowstyle}}
\newcommand{\rowstyle}[1]{\gdef\currentrowstyle{#1}%
  #1\ignorespaces
}

%%%%%%%%%%%%%%%%%%%%%%%
% Prepare title page and generate slide later %
%%%%%%%%%%%%%%%%%%%%%%%

\title{Template}
\subtitle{Examples slides}
\author{Jakob Everding} %\\ \footnotesize (Universität Hamburg)}
\institute{Universit\"at Hamburg}
%\titlegraphic{\includegraphics[width=3.5cm]{some_logo.jpg}}

\date{\today}

%%%%%%%%%%%%
% Document %
%%%%%%%%%%%%

\begin{document}
	\nocite{*}
%%%%%%%%%%%%%%%%%%%%%%%%%%%%%%%%%
%Gen. title page:
\usebackgroundtemplate{%             declare watermark
%\tikz[overlay,remember picture] \node[opacity=0.3, at=(current page.south east)] {
\tikz[overlay,remember picture] \node[opacity=1.0, at=(current page.center), xshift=3.8cm, yshift=-2.25cm] {
   \includegraphics[width=0.4\paperwidth]{output/figures/UHH-Siegel.jpg}};
}
\frame{\titlepage
\thispagestyle{empty}}
\usebackgroundtemplate{ }    %% undeclare watermark
%%%%%%%%%%%%%%%%%%%%%%%%%%%%%%%%%
%\tableofcontents
%\thispagestyle{empty}
\addtocounter{framenumber}{-1}
%%%%%%%%%%%%%%%%%%%%%%%%%%%%%%%%%

\section{Examples}
\subsection{}
\begin{frame}[label=ex_slide]
\frametitle{\hyperlink{introapp}{Example slide}}
\begin{enumerate}
	\item One
    \pause
	\item Two
	\begin{itemize}
		\item A
		\item B
	\end{itemize}
\end{enumerate}
\vspace{0.5cm}
\hyperlink{ex_table}{\beamergotobutton{Table: Example table}} \\
\end{frame}

\begin{frame}[label=ex_slide]
\frametitle{\hyperlink{introapp}{Example graph}}
\begin{figure}[firststage]
	\vspace{-0.2cm}
	\centering
	\caption{Open access logo.}
	\includegraphics[height=0.3\textheight]{output/figures/open_access_logo.png}	%trim: left bottom right top
\end{figure}
\vspace{-1cm}
\end{frame}

\miniframesoff
%% This command causes that for every following frame no additional bullet point appears in the header. \miniframeson \miniframesoff can be used as a switch turning it on and off.

\section{}
\begin{frame}[noframenumbering,plain]
\frametitle{Thank you}
\centering{Jakob Everding} \\
\vspace{0.5cm}
\centering{\footnotesize{Hamburg Center for Health Economics \\
	University of Hamburg \\
	\url{jakob.everding@uni-hamburg.de}
}}
\end{frame}

%% The following lines are creating the footer with its content and WITHOUT page numbers.
\showpagenumbersfalse


\section{}

\makeatletter
    \setbeamertemplate{headline}[default]
    \def\beamer@entrycode{\vspace*{-\headheight}}
\makeatother
\begin{frame}[label=ex_table,noframenumbering]
\frametitle{\hyperlink{introapp}{Example table}}
\begin{table} % h: here; t: top; b: bottom; p: own page
\vspace{-0.5cm}
\scriptsize %% Necessary for reducing the size of the table that it fits on the slide.
\setlength\tabcolsep{3.5pt}
\centering
\begin{threeparttable}
\caption{Header.} \label{tab:header}
\begin{tabular}{l *{3}{c}}
\hline
\addlinespace \multicolumn{1}{l}{Country} & \multicolumn{2}{c}{Dates} \\
\cmidrule(lr){2-3}
&\multicolumn{1}{c}{A}&\multicolumn{1}{c}{B} \\
\hline
\addlinespace
Austria&         1&         2\\
Australia&         3&         4\\
\hline
\end{tabular}
\begin{tablenotes}
\item \textit{Source:} ...
\end{tablenotes}
\end{threeparttable}
\end{table}
\vspace{0.5cm}
\hyperlink{ex_slide}{\beamergotobutton{Back}} \\
\end{frame}

\end{document}
